% Options for packages loaded elsewhere
\PassOptionsToPackage{unicode}{hyperref}
\PassOptionsToPackage{hyphens}{url}
%
\documentclass[
]{article}
\usepackage{amsmath,amssymb}
\usepackage{iftex}
\ifPDFTeX
  \usepackage[T1]{fontenc}
  \usepackage[utf8]{inputenc}
  \usepackage{textcomp} % provide euro and other symbols
\else % if luatex or xetex
  \usepackage{unicode-math} % this also loads fontspec
  \defaultfontfeatures{Scale=MatchLowercase}
  \defaultfontfeatures[\rmfamily]{Ligatures=TeX,Scale=1}
\fi
\usepackage{lmodern}
\ifPDFTeX\else
  % xetex/luatex font selection
\fi
% Use upquote if available, for straight quotes in verbatim environments
\IfFileExists{upquote.sty}{\usepackage{upquote}}{}
\IfFileExists{microtype.sty}{% use microtype if available
  \usepackage[]{microtype}
  \UseMicrotypeSet[protrusion]{basicmath} % disable protrusion for tt fonts
}{}
\makeatletter
\@ifundefined{KOMAClassName}{% if non-KOMA class
  \IfFileExists{parskip.sty}{%
    \usepackage{parskip}
  }{% else
    \setlength{\parindent}{0pt}
    \setlength{\parskip}{6pt plus 2pt minus 1pt}}
}{% if KOMA class
  \KOMAoptions{parskip=half}}
\makeatother
\usepackage{xcolor}
\usepackage[margin=1in]{geometry}
\usepackage{color}
\usepackage{fancyvrb}
\newcommand{\VerbBar}{|}
\newcommand{\VERB}{\Verb[commandchars=\\\{\}]}
\DefineVerbatimEnvironment{Highlighting}{Verbatim}{commandchars=\\\{\}}
% Add ',fontsize=\small' for more characters per line
\usepackage{framed}
\definecolor{shadecolor}{RGB}{248,248,248}
\newenvironment{Shaded}{\begin{snugshade}}{\end{snugshade}}
\newcommand{\AlertTok}[1]{\textcolor[rgb]{0.94,0.16,0.16}{#1}}
\newcommand{\AnnotationTok}[1]{\textcolor[rgb]{0.56,0.35,0.01}{\textbf{\textit{#1}}}}
\newcommand{\AttributeTok}[1]{\textcolor[rgb]{0.13,0.29,0.53}{#1}}
\newcommand{\BaseNTok}[1]{\textcolor[rgb]{0.00,0.00,0.81}{#1}}
\newcommand{\BuiltInTok}[1]{#1}
\newcommand{\CharTok}[1]{\textcolor[rgb]{0.31,0.60,0.02}{#1}}
\newcommand{\CommentTok}[1]{\textcolor[rgb]{0.56,0.35,0.01}{\textit{#1}}}
\newcommand{\CommentVarTok}[1]{\textcolor[rgb]{0.56,0.35,0.01}{\textbf{\textit{#1}}}}
\newcommand{\ConstantTok}[1]{\textcolor[rgb]{0.56,0.35,0.01}{#1}}
\newcommand{\ControlFlowTok}[1]{\textcolor[rgb]{0.13,0.29,0.53}{\textbf{#1}}}
\newcommand{\DataTypeTok}[1]{\textcolor[rgb]{0.13,0.29,0.53}{#1}}
\newcommand{\DecValTok}[1]{\textcolor[rgb]{0.00,0.00,0.81}{#1}}
\newcommand{\DocumentationTok}[1]{\textcolor[rgb]{0.56,0.35,0.01}{\textbf{\textit{#1}}}}
\newcommand{\ErrorTok}[1]{\textcolor[rgb]{0.64,0.00,0.00}{\textbf{#1}}}
\newcommand{\ExtensionTok}[1]{#1}
\newcommand{\FloatTok}[1]{\textcolor[rgb]{0.00,0.00,0.81}{#1}}
\newcommand{\FunctionTok}[1]{\textcolor[rgb]{0.13,0.29,0.53}{\textbf{#1}}}
\newcommand{\ImportTok}[1]{#1}
\newcommand{\InformationTok}[1]{\textcolor[rgb]{0.56,0.35,0.01}{\textbf{\textit{#1}}}}
\newcommand{\KeywordTok}[1]{\textcolor[rgb]{0.13,0.29,0.53}{\textbf{#1}}}
\newcommand{\NormalTok}[1]{#1}
\newcommand{\OperatorTok}[1]{\textcolor[rgb]{0.81,0.36,0.00}{\textbf{#1}}}
\newcommand{\OtherTok}[1]{\textcolor[rgb]{0.56,0.35,0.01}{#1}}
\newcommand{\PreprocessorTok}[1]{\textcolor[rgb]{0.56,0.35,0.01}{\textit{#1}}}
\newcommand{\RegionMarkerTok}[1]{#1}
\newcommand{\SpecialCharTok}[1]{\textcolor[rgb]{0.81,0.36,0.00}{\textbf{#1}}}
\newcommand{\SpecialStringTok}[1]{\textcolor[rgb]{0.31,0.60,0.02}{#1}}
\newcommand{\StringTok}[1]{\textcolor[rgb]{0.31,0.60,0.02}{#1}}
\newcommand{\VariableTok}[1]{\textcolor[rgb]{0.00,0.00,0.00}{#1}}
\newcommand{\VerbatimStringTok}[1]{\textcolor[rgb]{0.31,0.60,0.02}{#1}}
\newcommand{\WarningTok}[1]{\textcolor[rgb]{0.56,0.35,0.01}{\textbf{\textit{#1}}}}
\usepackage{graphicx}
\makeatletter
\def\maxwidth{\ifdim\Gin@nat@width>\linewidth\linewidth\else\Gin@nat@width\fi}
\def\maxheight{\ifdim\Gin@nat@height>\textheight\textheight\else\Gin@nat@height\fi}
\makeatother
% Scale images if necessary, so that they will not overflow the page
% margins by default, and it is still possible to overwrite the defaults
% using explicit options in \includegraphics[width, height, ...]{}
\setkeys{Gin}{width=\maxwidth,height=\maxheight,keepaspectratio}
% Set default figure placement to htbp
\makeatletter
\def\fps@figure{htbp}
\makeatother
\setlength{\emergencystretch}{3em} % prevent overfull lines
\providecommand{\tightlist}{%
  \setlength{\itemsep}{0pt}\setlength{\parskip}{0pt}}
\setcounter{secnumdepth}{-\maxdimen} % remove section numbering
\ifLuaTeX
  \usepackage{selnolig}  % disable illegal ligatures
\fi
\usepackage{bookmark}
\IfFileExists{xurl.sty}{\usepackage{xurl}}{} % add URL line breaks if available
\urlstyle{same}
\hypersetup{
  pdftitle={Power Law},
  pdfauthor={Berrie, Michel, Galuppini, Laurent},
  hidelinks,
  pdfcreator={LaTeX via pandoc}}

\title{Power Law}
\author{Berrie, Michel, Galuppini, Laurent}
\date{2025-03-07}

\begin{document}
\maketitle

\#1. Résultats théoriques :

On étudie le processus de Weibull (power law process dans la
terminologie américaine). C'est un processus de Poisson d'intensité :
\(\lambda(s)=\frac{\beta}{\alpha}(\frac{s}{\alpha})^{\beta-1}\) où les
paramètres \(\alpha\) et \(\beta\) sont inconnus.

\#\#1. 1 Estimateurs des paramètres inconnus :

Nous allons estimer les paramètres \(\alpha\) et \(\beta\) par leurs
estimateurs de maximum vraisemblance (MLE), \(\hat{\alpha}\) et
\(\hat{\beta}\).

On note \(\hat {\theta}\)=(\(\hat{\alpha}\),\(\hat{\beta}\).) tel que :

\(\hat{\theta} \in \arg\max_{\theta \in \mathbb{R}^+}  L(N,\theta)\)

D'après la proposition 4.14 du cours, la vraisemblance vaut dans le cas
d'un processus inhomogène :

\(L((N_t)_{t \in[0,T]; \theta})\) =
\(\left( \prod_{i=1}^{N_t}\lambda(T_i) \right) \exp \left(-\int_0^T \lambda(x)dx \right)\)

On remplace \(\lambda\) par la fonction du Weibull ainsi :

écrire l integrale de la feuille d'ema

Nous passons au log pour calculer la logvraisemblance (nous travaillons
sur R+) : =\textgreater{} justifier que on travail sur R+

\(l((N_t)_{t \in[0,T]; \theta})=log(L((N_t)_{t \in[0,T]; \theta}))\) =
\(N_t((log(\beta)-\beta log(\alpha))+ (\beta-1) \sum_{i=1}^{N_t}log(T_i)-(\frac{t}{\alpha})^\beta\)

Ensuite nous allons calculer le gradient de la logvraisemblance pour
trouver les estimateurs qui maximisent la log vraisemblance.

=\textgreater{} \(\nabla\)

Pour vérifier que nous sommes bien sur un maximum (vérification de la
concavité) nous allons analyser le signe de la Hessienne localement en
ce point :

=\textgreater gradient, hessienne locale pour verifier la concavité

Finalement on a bien
\(\hat{\theta} \in \arg\max_{\theta \in \mathbb{R}^+}  L(N,\theta)\) et
nos estimateurs sont bien des estimateurs de maximum vraisemblance.

Ces estimateurs vont nous permettre de travailler ensuite sur la
construction d'intervalle de confiance de nos paramètres inconnus.

C'est ce que nous allons faire ci-dessous.

\#\#1. 2. Intervalle de confiance et loi

=\textgreater{} César et Florian

\#Résultats numériques

Second, the students are asked to perform numerical simulations in order
to validate, from a practical point of view, the confidence intervals
presented in {[}Cocozza-Thivent, 1997{]},

Simulation of a homogeneous Poisson process with intensity
\texttt{lambda} on the window {[}0,\texttt{Tmax}{]}.

\begin{Shaded}
\begin{Highlighting}[]
\NormalTok{simulPPh1 }\OtherTok{\textless{}{-}} \ControlFlowTok{function}\NormalTok{(lambda,Tmax)}
\NormalTok{\{ n}\OtherTok{=}\FunctionTok{rpois}\NormalTok{(}\DecValTok{1}\NormalTok{,lambda}\SpecialCharTok{*}\NormalTok{Tmax)}
\NormalTok{  u}\OtherTok{=}\FunctionTok{runif}\NormalTok{(n,}\DecValTok{0}\NormalTok{,Tmax)}
\NormalTok{  p}\OtherTok{=}\FunctionTok{sort}\NormalTok{(u)}
  \FunctionTok{return}\NormalTok{(p)}
\NormalTok{\}}
\end{Highlighting}
\end{Shaded}

Simulation d'un processus de poisson inhomogène :

\begin{Shaded}
\begin{Highlighting}[]
\NormalTok{simulPPi }\OtherTok{=} \ControlFlowTok{function}\NormalTok{(lambda\_fct,Tmax,M)}
\NormalTok{\{n}\OtherTok{=}\FunctionTok{cumsum}\NormalTok{(}\FunctionTok{lambda\_fct}\NormalTok{(}\DecValTok{0}\SpecialCharTok{:}\NormalTok{Tmax))}
\NormalTok{  t}\OtherTok{=} \FunctionTok{simulPPh1}\NormalTok{(M,Tmax)}
\NormalTok{  y}\OtherTok{=}\FunctionTok{runif}\NormalTok{(t,}\DecValTok{0}\NormalTok{,M)}
\NormalTok{  t[y}\SpecialCharTok{\textless{}=}\FunctionTok{lambda\_fct}\NormalTok{(t)]}
  \FunctionTok{return}\NormalTok{(t)}
\NormalTok{\}}
\end{Highlighting}
\end{Shaded}

Let us define a plot function for a counting process \texttt{PP}.

\begin{Shaded}
\begin{Highlighting}[]
\NormalTok{plot.PP}\OtherTok{\textless{}{-}} \ControlFlowTok{function}\NormalTok{(PP)}
\NormalTok{\{}
  \CommentTok{\# plot the counting process (with jumps of size 1 (starting at point (0,0))):}
  \FunctionTok{plot}\NormalTok{(}\FunctionTok{c}\NormalTok{(}\DecValTok{0}\NormalTok{,PP),}\DecValTok{0}\SpecialCharTok{:}\FunctionTok{length}\NormalTok{(PP),}\AttributeTok{type=}\StringTok{"s"}\NormalTok{,}\AttributeTok{xlab=}\StringTok{"time t"}\NormalTok{,}\AttributeTok{ylab=}\StringTok{"number of events by time t"}\NormalTok{)}
  \CommentTok{\# add the arrival times on the horizontal axis: }
  \FunctionTok{points}\NormalTok{(PP,}\DecValTok{0}\SpecialCharTok{*}\NormalTok{PP,}\AttributeTok{type=}\StringTok{"p"}\NormalTok{,}\AttributeTok{pch=}\DecValTok{16}\NormalTok{)}
  \CommentTok{\# link arrival times with the counts:}
  \FunctionTok{lines}\NormalTok{(PP,}\DecValTok{0}\SpecialCharTok{:}\NormalTok{(}\FunctionTok{length}\NormalTok{(PP)}\SpecialCharTok{{-}}\DecValTok{1}\NormalTok{),}\AttributeTok{type=}\StringTok{"h"}\NormalTok{,}\AttributeTok{lty=}\DecValTok{2}\NormalTok{)}
\NormalTok{\}}
\end{Highlighting}
\end{Shaded}

\begin{Shaded}
\begin{Highlighting}[]
\NormalTok{Tmax}\OtherTok{=}\DecValTok{1000}
\NormalTok{alpha }\OtherTok{=} \DecValTok{1}
\NormalTok{beta}\OtherTok{=} \DecValTok{1}
\NormalTok{lambda\_fct }\OtherTok{\textless{}{-}} \ControlFlowTok{function}\NormalTok{(x)\{}\FunctionTok{return}\NormalTok{(alpha}\SpecialCharTok{/}\NormalTok{beta}\SpecialCharTok{*}\NormalTok{(x}\SpecialCharTok{/}\NormalTok{alpha)}\SpecialCharTok{\^{}}\NormalTok{(beta}\DecValTok{{-}1}\NormalTok{))\}}
\NormalTok{M1}\OtherTok{=}\DecValTok{10}
\NormalTok{PPi1 }\OtherTok{=} \FunctionTok{simulPPi}\NormalTok{(lambda\_fct,Tmax,M1)}
\FunctionTok{par}\NormalTok{(}\AttributeTok{mfrow=}\FunctionTok{c}\NormalTok{(}\DecValTok{1}\NormalTok{,}\DecValTok{2}\NormalTok{))}
\FunctionTok{curve}\NormalTok{(lambda\_fct,}\AttributeTok{from=}\DecValTok{0}\NormalTok{,}\AttributeTok{to=}\NormalTok{Tmax,}\AttributeTok{n=}\DecValTok{1000}\NormalTok{)}
\FunctionTok{plot.PP}\NormalTok{(PPi1)}
\end{Highlighting}
\end{Shaded}

\#\#estimateurs de alpha et beta MLE

\begin{Shaded}
\begin{Highlighting}[]
\CommentTok{\#estimalpha=T*exp({-}1/beta*log(simulPPi()))}

\CommentTok{\#il y a un message d\textquotesingle{}erreur car il manque les arguement dans la fonction simulPPi}
\end{Highlighting}
\end{Shaded}


\end{document}
